\documentclass{article}
\usepackage[brazil]{babel}
\usepackage{graphicx}
\usepackage{graphics}
\usepackage{float}
\usepackage{listings}
\title{EP4:\\ Integração de Monte Carlo por cadeias de Markov}
\author{José Paulo Silva Cavalcante}
\date{\today}
\begin{document}
\maketitle
\section{Introdução}
O Método MCMC(Markov Chain Monte Carlo) consiste na simulação de uma função distribuição de probabilidade $f(x)$ através da construção de uma cadeia de Markov de modo que a distribuição estacionária seja dada por $\pi(x) = f(x) $.
\section{Considerações Iniciais}

Estamos interessados em estimar o z, onde:\\

\begin{equation}
 z = \int_{0}^{\infty}h(x)g(x)dx
\end{equation}
\\

\noindent por sua vez,
$h(x)=I(1<x<2)$ e $g(x) \propto gamma(C,x)|cos(Rx)|$ onde gamma tem distribuição exponencial(C), para C = 1.373003882 e R = 1.10297373.\\
Vamos implementar o MCMC com o Núcleo Normal de média 0 e variância $s^2$.\\
Utilizaremos duas probabilidades de aceitação, chamadas de $\alpha$:\\\\
Metrópolis Hasting:\\
\begin{equation}
\alpha(x^{(k)},x^{(k-1)}) = min\Bigg(1,\frac{g(x^{(k)})}{g(x^{(k-1)})}\Bigg)
\end{equation}
Barker:
\begin{equation}
\alpha(x^{(k)},x^{(k-1)}) = min\Bigg(1,\frac{g(x^{(k)})}{g(x^{(k-1)})+g(x^{(k)})}\Bigg)
\end{equation}
Seja $f(x)= gamma(C,x)|cos(Rx)|$, temos que $g(x) = a\cdot gamma(C,x)|cos(Rx)|$, logo, neste caso teremos as seguintes probabilidades de aceitação:\\
Metropolis:
\begin{equation}
\alpha(x^{(k)},x^{(k-1)}) = min\Bigg(1,\frac{a\cdot f(x^{(k)})}{a\cdot f(x^{(k-1)})}\Bigg)=min\Bigg(1,\frac{f(x^{(k)})}{f(x^{(k-1)})}\Bigg)	
\end{equation}
Barker:
\begin{equation}
\alpha(x^{(k)},x^{(k-1)}) = min\Bigg(1,\frac{a\cdot f(x^{(k)})}{a\cdot f(x^{(k-1)})+a\cdot f(x^{(k)})}\Bigg)=min\Bigg(1,\frac{f(x^{(k)})}{f(x^{(k-1)})+f(x^{(k)})}\Bigg)	
\end{equation}
Logo, podemos observar que as duas probabilidades de aceitação não possuem a constante de integração a, portanto, não precisamos saber a expressão da função g(x), precisamos somente de uma função proporcional a ela: f(x).

\subsection{Algoritmo}
\begin{lstlisting}[language=R]

aceita_metropolis <- function(x_inicial, s){
x_proposto <- rnorm(1,x_inicial,s)
alpha <- min(1, f(x_proposto)/f(x_inicial))
if (runif(1) < alpha)
x_inicial <- x_proposto
return (x_inicial)
}

aceita_barker <- function(x_inicial,s){
x_proposto <- rnorm(1,x_inicial,s)
alpha <- f(x_proposto)/(f(x_inicial) + f(x_proposto))
if (runif(1) < alpha)
x_inicial <- x_proposto
return (x_inicial)
}
\end{lstlisting}
\section{MCMC}
O algoritmo de MCMC segue os seguintes passos:\\
1. Especifique um valor inicial $x^{(0)}$(k=1) e $s^2$\\
2. Gere uma amostra $x^{k+1}$ com distribuição Normal($x^{(k)}$,$s$)\\
3. Calcule a probabilidade de aceitação (utilizando (2) ou (3))\\
4. Gere uma u ~ Uniforme(0,1)\\
5. Se u $\leq$ $\alpha(x^{(k)},x^{(k+1)})$: a cadeia recebe $x^{(k+1)}$ e $x^{(k)}$  = $x^{(k+1)}$ , caso contrário, recebe $x^{(k)}$\\
Volte para o passo (2), terminamos após n iterações.\\
Podemos ver que devido as expressões só serão aceitas amostras positivas, pois, ao inserirmos um valor negativo em ambos os alfas, temos zero como resultado, o que é sempre menor do que a amostra gerada pela Uniforme, nos permitindo ficar sempre dentro do domínio de integração. 

\begin{figure}[h]
	\begin{center}
		\includegraphics[scale=0.6]{amostra_metropolis}
		\caption{Comportamento do histograma em relação a g(x)}
	\end{center}
\end{figure}

\subsection{Algoritmo}
\begin{lstlisting}[language = R]

amostra_metropolis <- function(x, s, n){
amostra <- numeric(n)
for (i in seq_len(n)){
amostra[i] <- x <- aceita_metropolis(x,s)
}
return(amostra)
}

amostra_barker <- function(x, s, n){
amostra <- numeric(n)
rejeitados <- FALSE
for (i in seq_len(n)){
amostra[i] <- x <- aceita_barker(x,s)
}
return(amostra)
}
\end{lstlisting}
\section{Aquecimento da cadeia}
Para utilizarmos a cadeia gerada pelo MCMC, precisamos descartar uma faixa inicial de pontos, isso se justifica pelo fato que no passo (2) utilizamos o ponto anterior como média da Normal, logo, não temos independência entre os pontos gerados. Porém, conforme o crescimento do número de iterações, a cadeia começa a ter comportamento Markoviano, onde somente o ponto anterior tem influência no novo a ser gerado. Ou seja, queremos descartar os pontos iniciais que possuem uma alta correlação linear. Não há consenso na literatura sobre a porcentagem de pontos a serem descartados na cadeia, vamos inicialmente trabalhar com a exclusão de 30\% (serão eliminados do início da cadeia), a escolha foi baseada no gráfico de auto correlação:\\

\begin{figure}[h]
\begin{center}
	\includegraphics[scale=0.6]{burn}
	\caption{Efeito do Burn-in nas autocorrelações}
\end{center}
\end{figure}
Podemos observar que há uma redução nas auto correlações das amostras ao retirar 30\% dos pontos iniciais, quando aumentamos a porcentagem não há mudança significativa no gráfico gerado.\\

\begin{figure}[H]
	\begin{center}
		\includegraphics[scale=0.6]{burn2}
		\caption{Efeito do Burn-in na cadeia de Markov}
	\end{center}
\end{figure}

Para ilustrar melhor o efeito do burn-in, definimos $x_0$ = 10, logo, no gráfico acima podemos ver que graças à retirada do ponto inicial o burn-in nos oferece uma cadeia mais estável, independentemente do ponto inicial adotado(que pode ser discrepante em relação à amostra esperada).

\section{Calibragem do s}
Como antes dito, o núcleo do nosso MCMC tem distribuição Normal($x^{(k)},s$), no ultimo tópico vimos que podemos definir qualquer ponto $x^{(0)}$ devido ao burn-in, porém, o mesmo não vale para o s, já que temos que escolher um desvio padrão que não permita grandes saltos dentro da cadeia de Markov. Podemos utilizar a proporção de pontos aceitos para cada alfa como ilustração do tamanho do salto da cadeia, adotamos como ideal uma taxa de pelo menos 50\%.
\section{Estimação de z}
Para estimarmos o valor de z, utilizaremos o Teorema do Limite Central, assim como nos EPs anteriores,porém, devemos assumir que todas as amostras presentes na cadeia são independentes(o que não ocorre), além disso, como a função $h(x)$ é uma indicadora para o intervalo (1,2), $\widehat{z}$ será dado por:
\begin{equation}
\widehat{z}=\frac{\sum_{0}^{n}h(x_i)}{n}
\end{equation}
Onde n é a quantidade de pontos $x_i$ da cadeia.
\subsection{Algoritmo}
\begin{lstlisting}[language=R]
estimativa <- function(x){sum(h(x))/length(x)}
\end{lstlisting}
\section{Erro}
Levando em consideração que queremos um erro de 1\% podemos estimar o erro através do intervalo de confiança para $\widehat{z}$ e de acordo com o TLC, temos:
\begin{equation}
e = Z(0.99)\cdot\sqrt{\frac{\sigma}{n}}
\end{equation}
Para calcularmos um erro relativo de 1\% podemos repetir a cadeia M vezes e calcular a esperança das estimativas de todas elas.
\section{Implementação}
Função que recebe o x inicial, desvio padrão(s),tamanho da cadeia (n) e porcentagem de descarte para o aquecimento\\
Retorna as estimativas, erros e proporções de aceitos para cada tipo de rejeição.
\begin{lstlisting}[language=R]

estima_b_m <- function(x0,s,n,p){
n2 <- n*p
amostra_m <- amostra_metropolis(x0,s,n)
amostra_b <- amostra_barker(x0,s,n)
library(dplyr)
prop_aceitos_m <- n_distinct(amostra_m)/length(amostra_m)
prop_aceitos_b <- n_distinct(amostra_b)/length(amostra_b)
burn_m <- amostra_m[n2:n]
burn_b <- amostra_b[n2:n]
erro_m <- dnorm(0.99)*sqrt(var(burn_m)/length(burn_m))
erro_b <- dnorm(0.99)*sqrt(var(burn_b)/length(burn_b))
return(list(estimativa_metropolis = estimativa(burn_m),
erro_metropolis = erro_m,proporcao_metropolis =prop_aceitos_m,
estimativa_barker = estimativa(burn_b),erro_barker=erro_b,
 proporcao_barker = prop_aceitos_b))
\end{lstlisting}
\section{Simulações}
Simulamos para os seguintes valores:\\
\begin{lstlisting}
primeira tentativa
x0 = 1, s = 1, n  = 10000,p =0.2
estima_b_m(1,1,10000,0.2)
estimativa_metropolis
[1]0.06849144

erro_metropolis
[1] 0.001691854

proporcao_metropolis
[1] 0.3436

estimativa_barker
[1] 0.06211724

erro_barker
[1] 0.001907863

proporcao_barker
[1] 0.2161

podemos ver que nessa tentativa houve uma pequena
porcentagem de aceitacao para os dois metodos
vamos reduzir a variancia para melhorar essa porcentagem:

segunda tentativa

x0 = 1, s = 0.5, n  = 10000,p =0.2
estima_b_m(1,0.5,10000,0.2)

estimativa_metropolis
[1] 0.06874141

erro_metropolis
[1] 0.002044394

proporcao_metropolis
[1] 0.5136

estimativa_barker
[1] 0.06349206

erro_barker
[1] 0.00155439

proporcao_barker
[1] 0.3041
vemos uma boa proporcao para o metodo de metropolis,
mas ainda temos uma baixa proporcao em Barker

terceira tentativa
desta vez, vamos descartar 30% dos pontos gerados,
devido ao burn-in
x0 = 1, s = 0.2, n  = 10000,p =0.3
estima_b_m(1,0.2,10000,0.3)
estimativa_metropolis
[1] 0.0659562

erro_metropolis
[1] 0.0007491115

proporcao_metropolis
[1] 0.749

estimativa_barker
[1] 0.06128484

erro_barker
[1] 0.0005599133

proporcao_barker
[1] 0.42121
\end{lstlisting}
\section{Resultados}
As estimativas e os erros apresentados através de replicações foram:
\begin{figure}[h]
	\begin{center}
		\includegraphics{tabela}
		\caption{Resultados finais}
	\end{center}
\end{figure}

\section{Referências}
$[1]$ STERN, Julio Michael. Cognitive Construtivism and the Epistemic Significance of Sharp Statistical Hypotheses in Natural Sciences IME-USP, 2012. Disponível em : https:$//$www.ime.usp.br/~jstern/books/evli.pdf. Acesso em 2 abr. 2020.

\end{document}